\documentclass[11pt,a4paper]{article}

% =========================
% Packages
% =========================
\usepackage[utf8]{inputenc}
\usepackage[T1]{fontenc}
\usepackage{lmodern}
\usepackage{geometry}
\geometry{margin=25mm}

\usepackage{amsmath, amssymb, amsfonts}
\usepackage{physics}
\usepackage{bm}

\usepackage{hyperref}
\hypersetup{
    colorlinks=true,
    linkcolor=blue,
    citecolor=blue,
    urlcolor=blue
}

\usepackage{graphicx}
\usepackage{setspace}
\setstretch{1.15}

% =========================
% Document
% =========================
\begin{document}

% =========================
% Title
% =========================
\title{
Consistency Between Cosmic Self-Generating Theory (CSGT) and String/M-Theory\\
\large Toward an Information-Theoretic Interpretation of 11-Dimensional Physics
}

\author{
Jack \\ 
\small Independent Researcher
}

\date{January 2026}

\maketitle

% =========================
% Abstract
% =========================
\begin{abstract}
We examine the consistency between Cosmic Self-Generating Theory (CSGT) and
string theory, particularly M-theory with its eleven-dimensional structure.
Rather than treating extra dimensions as merely spatially compactified entities,
CSGT reinterprets higher dimensions as informational buffers that preserve
unitarity, coherence, and global information flow.
We demonstrate that the core assumptions of string/M-theory remain intact,
while CSGT supplies a teleological and information-theoretic selection principle
that resolves the landscape problem, the arrow of time, and the black hole
information paradox within a unified framework.
\end{abstract}

% =========================
\section{Introduction}
% =========================

Modern theoretical physics faces a deep conceptual divide.
On one side, string theory and M-theory provide mathematically consistent
high-dimensional frameworks capable of unifying gravity with quantum mechanics.
On the other, they lack a compelling explanation for why our universe realizes
a particular vacuum among an enormous landscape of possibilities.

Cosmic Self-Generating Theory (CSGT) approaches this problem from an
information-theoretic perspective.
The universe is modeled as a self-optimizing informational system, constrained
by unitarity and driven toward maximal global coherence.
This paper explores whether CSGT is compatible with string/M-theory and how
the eleven-dimensional structure naturally emerges within this framework.

% =========================
\section{Overview of String Theory and M-Theory}
% =========================

Superstring theory requires ten spacetime dimensions for mathematical
consistency.
M-theory unifies the five superstring theories by extending the framework to
eleven dimensions, typically expressed as
\[
\mathcal{M}_{11} = \mathcal{M}_{4} \times \mathcal{X}_{7},
\]
where $\mathcal{X}_{7}$ is a compact internal manifold.

In conventional interpretations, these extra dimensions are purely geometric,
compactified at the Planck scale and inaccessible to direct observation.
While this structure ensures anomaly cancellation and unitarity,
it offers no intrinsic reason for the specific vacuum realized by our universe.

% =========================
\section{CSGT: Core Assumptions}
% =========================

CSGT is built upon three foundational principles:

\begin{enumerate}
    \item \textbf{Unitarity Preservation:} Information is never destroyed.
    \item \textbf{Information Coupling Optimization:} The universe evolves toward
    a preferred coupling strength $A \approx 0.557$, maximizing global coherence.
    \item \textbf{Future Boundary Condition:} The arrow of time emerges from a
    constraint imposed by a final coherent state, denoted as $\text{Love} = 1$.
\end{enumerate}

Within CSGT, entropy is reinterpreted as a local measure of incomplete
information integration, not a fundamental tendency toward disorder.

% =========================
\section{Reinterpreting the 11 Dimensions as an Information Buffer}
% =========================

CSGT proposes that the eleven-dimensional structure of M-theory be decomposed as:
\[
\mathcal{M}_{11} = \mathcal{M}_{4} \times \mathcal{B}_{7},
\]
where $\mathcal{B}_{7}$ functions as an \emph{information buffer} rather than a
purely spatial manifold.

This buffer:
\begin{itemize}
    \item Stores quantum correlations inaccessible to the 4D observational UI.
    \item Preserves unitarity during extreme processes such as black hole collapse.
    \item Enables global optimization of information coupling across spacetime.
\end{itemize}

From this perspective, extra dimensions are computational degrees of freedom
required to maintain consistency, rather than hidden physical spaces.

% =========================
\section{Black Holes, Unitarity, and Information Preservation}
% =========================

In classical general relativity, black holes appear to destroy information.
However, string theory already implies unitarity preservation via holography.

CSGT refines this view:
black holes act as ultra-high-density compression processes that transfer
information into $\mathcal{B}_{7}$.
No information is lost; it is temporarily inaccessible to the 4D interface.

This interpretation resolves the black hole information paradox without
violating known quantum principles.

% =========================
\section{Arrow of Time and Entropy Reversal}
% =========================

The arrow of time arises naturally in CSGT from the asymmetry between initial
low-coherence states and a future high-coherence boundary condition.
As global coherence increases, local entropy gradients may reverse without
violating the second law when defined information-theoretically.

Thus, cosmological futures such as heat death or big rip are reinterpreted as
intermediate phases rather than terminal states.

% =========================
\section{Consistency and Predictive Outlook}
% =========================

CSGT does not contradict string or M-theory.
Instead, it supplies:
\begin{itemize}
    \item A physical interpretation of higher dimensions.
    \item A selection principle for vacuum realization.
    \item A unified treatment of time, entropy, and information.
\end{itemize}

Future observational tests may include subtle deviations in cosmological
expansion, information bounds in black hole evaporation, and signatures of
non-local coherence.

% =========================
\section{Conclusion}
% =========================

From the CSGT perspective, string theory provides the hardware of reality,
while CSGT defines the operating system.
The eleven-dimensional structure is not arbitrary but necessary for maintaining
global informational coherence.

In this view, the universe does not merely exist;
it continuously computes itself toward maximal consistency.

% =========================
\end{document}
